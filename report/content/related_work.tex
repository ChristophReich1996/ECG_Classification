\section{Related Work} \label{sec:rleated_work}
Recent approaches for the task of AF classification in ECG recordings can be clustered in two groups. First, classical machine learning approaches \cite{Antink2017, Smisek2017}, and second, deep learning approaches \cite{Zihlmann2017, Mousavi2019, Mashrur2019, Khriji2020, Nonaka2020}. In general, deep learning approaches achieve better classification accuracy, however, sacrifice explainability \cite{Goodfellow2016, Zihlmann2017, Samek2018, Mousavi2019}.\\
\indent Classical machine learning approaches typically extract features, first and classify them in a section learnable step. Hoog Antink \etal \cite{Antink2017} proposed an approach that first extracts ECG timing features, robust interval features, and waveform features \cite{Antink2017}. All extracted features are fed into a learned random forest for classification \cite{Antink2017}. Other approaches perform similar feature extractions but utilize different learnable classification methods, such as support vector machines \cite{Smisek2017}. These approaches, however, require a lot of domain knowledge to extract relevant features. Additionally, hand-crafted feature extraction approaches are often complicated and error prune to implement.\\
\indent Recently deep learning-based approaches have been applied to AF classification in ECG recordings \cite{Zihlmann2017, Mousavi2019, Mashrur2019, Khriji2020, Nonaka2020}. These approaches typically produce a spectrogram of the ECG recording and feed it into a deep neural network for classification \cite{Zihlmann2017, Mashrur2019}. Deep learning AF classification approaches tend to outperform more classical machine learning approaches \cite{Mousavi2019, Khriji2020}. Using deep neural networks, however, require a reasonable amount of data, or extensive data augmentation \cite{Perez2017}, and has higher computational requirements \cite{Lecun2015, Goodfellow2016}. Since a deep neural network is able to learn the extraction of relevant features, heavy preprocessing is typically not required \cite{Zihlmann2017, Lecun2015, Goodfellow2016}.