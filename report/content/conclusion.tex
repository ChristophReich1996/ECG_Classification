\section{Conclusion} \label{sec:conclusion}

In this study we presented a novel deep neural network architecture for classifying arrhythmia's in single-lead ECG signals with variable lengths. The proposed ECG-DualNet(++) employs input data from both the time and frequency domain. In the task of atrial fibrillation classification, on the 2017 PhysioNet dataset, ECG-DualNet outperforms recent CNN approaches. Additionally, extensive pre-training on the Icentia$11$k dataset was performed. This pre-training does not lead to improvements in classification performance on the target dataset.
% The introduced ECG-DualNet utilizes input data in the frequency and time domain. The time-domain ECG signal gets processed by a separate encoder. The resulting encoded latent vector gets employed in the spectrogram encoder by conditional batch normalization. We also extended ECG-DualNet by the use of Transformers and Axial-Attentions (ECG-DualNet++). With the use of both input data from the time and frequency domain, we were able to outperform a CNN baseline which only utilizes input data from the frequency domain.