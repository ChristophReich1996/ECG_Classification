\section{Introduction}

\subsection{Problem Setting}
\begin{frame}[c]{Introduction}
\framesubtitle{Problem Setting}
    \begin{center}
        \begin{large}
            \textbf{Classify single-lead ECG signals with variable length.}
        \end{large}
    \end{center}
    \vspace{0.5cm}
    \begin{figure}[!ht]
        \setlength{\figH}{3.75cm}
        \setlength{\figW}{0.95\textwidth}
        \centering
        % This file was created by tikzplotlib v0.9.6.
\begin{tikzpicture}
\pgfplotsset{
   every axis/.append style={
		font=\fontsize{10}{10}\sffamily},
	every non boxed x axis/.append style={
		x axis line style={->}
	},
	every non boxed y axis/.append style={
		y axis line style={->}
	},
	every non boxed z axis/.append style={
		z axis line style={->}
	}
}
\begin{axis}[
height=\figH,
tick align=outside,
tick pos=left,
width=\figW,
x grid style={white!69.0196078431373!black},
xlabel={time/$\si{\second}$},
xmin=-0.208166670799255, xmax=4.37150008678436,
xtick style={color=black},
y grid style={white!69.0196078431373!black},
ylabel={voltage/$\si{\milli\volt}$},
ymin=-723.5, ymax=695.5,
ytick style={color=black},
ytick = {-500, 0, 500},
yticklabels = {-500, 0, 500},
xtick = {0, 1, 2, 3, 4},
xticklabels = {0, 1, 2, 3, 4},
]
\addplot [semithick, tud1b]
table {%
0 9
0.00333333341404796 37
0.00666666682809591 61
0.00999999977648258 89
0.0133333336561918 116
0.0166666675359011 137
0.0199999995529652 159
0.0233333334326744 190
0.0266666673123837 210
0.0299999993294477 216
0.0333333350718021 220
0.0366666652262211 224
0.0399999991059303 228
0.0433333329856396 233
0.0466666668653488 242
0.0500000007450581 257
0.0533333346247673 267
0.0566666685044765 274
0.0599999986588955 274
0.0633333325386047 266
0.0666666701436043 258
0.0700000002980232 249
0.0733333304524422 242
0.0766666680574417 224
0.0799999982118607 199
0.0833333358168602 168
0.0866666659712791 136
0.0900000035762787 106
0.0933333337306976 82
0.0966666638851166 67
0.100000001490116 53
0.103333331644535 40
0.106666669249535 31
0.109999999403954 25
0.113333337008953 19
0.116666667163372 12
0.119999997317791 2
0.123333334922791 -11
0.126666665077209 -25
0.129999995231628 -37
0.133333340287209 -53
0.136666670441628 -76
0.140000000596046 -90
0.143333330750465 -97
0.146666660904884 -101
0.150000005960464 -103
0.153333336114883 -106
0.156666666269302 -109
0.159999996423721 -112
0.16333332657814 -116
0.16666667163372 -124
0.170000001788139 -140
0.173333331942558 -160
0.176666662096977 -161
0.180000007152557 -132
0.183333337306976 -84
0.186666667461395 -45
0.189999997615814 -16
0.193333327770233 4
0.196666672825813 40
0.200000002980232 87
0.203333333134651 118
0.20666666328907 137
0.209999993443489 136
0.213333338499069 113
0.216666668653488 32
0.219999998807907 -51
0.223333328962326 -78
0.226666674017906 -92
0.230000004172325 -112
0.233333334326744 -145
0.236666664481163 -187
0.239999994635582 -201
0.243333339691162 -206
0.246666669845581 -199
0.25 -188
0.253333330154419 -177
0.256666660308838 -170
0.259999990463257 -166
0.263333320617676 -164
0.266666680574417 -166
0.270000010728836 -172
0.273333340883255 -179
0.276666671037674 -186
0.280000001192093 -175
0.283333331346512 -144
0.286666661500931 -81
0.28999999165535 18
0.293333321809769 162
0.296666651964188 326
0.300000011920929 448
0.303333342075348 469
0.306666672229767 400
0.310000002384186 281
0.313333332538605 135
0.316666662693024 -12
0.319999992847443 -137
0.323333323001862 -228
0.326666653156281 -293
0.330000013113022 -329
0.333333343267441 -343
0.33666667342186 -343
0.340000003576279 -335
0.343333333730698 -323
0.346666663885117 -311
0.349999994039536 -304
0.353333324193954 -300
0.356666654348373 -309
0.360000014305115 -321
0.363333344459534 -330
0.366666674613953 -332
0.370000004768372 -336
0.373333334922791 -331
0.376666665077209 -315
0.379999995231628 -302
0.383333325386047 -281
0.386666655540466 -254
0.389999985694885 -223
0.393333345651627 -196
0.396666675806046 -186
0.400000005960464 -181
0.403333336114883 -178
0.406666666269302 -179
0.409999996423721 -188
0.41333332657814 -197
0.416666656732559 -205
0.419999986886978 -209
0.423333346843719 -207
0.426666676998138 -204
0.430000007152557 -195
0.433333337306976 -168
0.436666667461395 -137
0.439999997615814 -106
0.443333327770233 -66
0.446666657924652 -20
0.449999988079071 22
0.453333348035812 54
0.456666678190231 70
0.46000000834465 82
0.463333338499069 97
0.466666668653488 111
0.469999998807907 121
0.473333328962326 121
0.476666659116745 113
0.479999989271164 102
0.483333319425583 89
0.486666679382324 81
0.490000009536743 76
0.493333339691162 70
0.496666669845581 63
0.5 40
0.503333330154419 16
0.506666660308838 1
0.509999990463257 -15
0.513333320617676 -28
0.516666650772095 -47
0.519999980926514 -69
0.523333311080933 -91
0.526666641235352 -111
0.529999971389771 -135
0.533333361148834 -162
0.536666691303253 -185
0.540000021457672 -194
0.543333351612091 -202
0.54666668176651 -208
0.550000011920929 -213
0.553333342075348 -218
0.556666672229767 -226
0.560000002384186 -235
0.563333332538605 -244
0.566666662693024 -251
0.569999992847443 -261
0.573333323001862 -271
0.576666653156281 -279
0.579999983310699 -284
0.583333313465118 -278
0.586666643619537 -267
0.589999973773956 -255
0.593333303928375 -245
0.596666693687439 -237
0.600000023841858 -232
0.603333353996277 -217
0.606666684150696 -192
0.610000014305115 -158
0.613333344459534 -117
0.616666674613953 -60
0.620000004768372 10
0.623333334922791 80
0.626666665077209 124
0.629999995231628 125
0.633333325386047 89
0.636666655540466 -9
0.639999985694885 -150
0.643333315849304 -309
0.646666646003723 -453
0.649999976158142 -559
0.653333306312561 -625
0.656666696071625 -650
0.660000026226044 -659
0.663333356380463 -659
0.666666686534882 -650
0.670000016689301 -624
0.673333346843719 -593
0.676666676998138 -583
0.680000007152557 -588
0.683333337306976 -604
0.686666667461395 -628
0.689999997615814 -638
0.693333327770233 -649
0.696666657924652 -619
0.699999988079071 -554
0.70333331823349 -467
0.706666648387909 -375
0.709999978542328 -297
0.713333308696747 -238
0.716666638851166 -195
0.720000028610229 -165
0.723333358764648 -140
0.726666688919067 -114
0.730000019073486 -91
0.733333349227905 -78
0.736666679382324 -66
0.740000009536743 -52
0.743333339691162 -27
0.746666669845581 0
0.75 21
0.753333330154419 35
0.756666660308838 48
0.759999990463257 58
0.763333320617676 66
0.766666650772095 83
0.769999980926514 117
0.773333311080933 158
0.776666641235352 190
0.779999971389771 211
0.783333361148834 223
0.786666691303253 242
0.790000021457672 269
0.793333351612091 284
0.79666668176651 297
0.800000011920929 305
0.803333342075348 301
0.806666672229767 288
0.810000002384186 267
0.813333332538605 253
0.816666662693024 239
0.819999992847443 223
0.823333323001862 194
0.826666653156281 159
0.829999983310699 125
0.833333313465118 100
0.836666643619537 76
0.839999973773956 50
0.843333303928375 34
0.846666693687439 8
0.850000023841858 -22
0.853333353996277 -47
0.856666684150696 -61
0.860000014305115 -73
0.863333344459534 -86
0.866666674613953 -105
0.870000004768372 -127
0.873333334922791 -139
0.876666665077209 -144
0.879999995231628 -147
0.883333325386047 -149
0.886666655540466 -153
0.889999985694885 -159
0.893333315849304 -165
0.896666646003723 -170
0.899999976158142 -173
0.903333306312561 -177
0.906666696071625 -179
0.910000026226044 -182
0.913333356380463 -183
0.916666686534882 -183
0.920000016689301 -182
0.923333346843719 -181
0.926666676998138 -181
0.930000007152557 -181
0.933333337306976 -180
0.936666667461395 -174
0.939999997615814 -152
0.943333327770233 -110
0.946666657924652 -52
0.949999988079071 0
0.95333331823349 47
0.956666648387909 88
0.959999978542328 104
0.963333308696747 101
0.966666638851166 71
0.970000028610229 2
0.973333358764648 -95
0.976666688919067 -202
0.980000019073486 -300
0.983333349227905 -379
0.986666679382324 -434
0.990000009536743 -476
0.993333339691162 -516
0.996666669845581 -554
1 -589
1.00333333015442 -613
1.00666666030884 -621
1.00999999046326 -615
1.01333332061768 -596
1.01666665077209 -563
1.01999998092651 -548
1.02333331108093 -543
1.02666664123535 -537
1.02999997138977 -526
1.03333330154419 -493
1.03666663169861 -428
1.03999996185303 -344
1.04333329200745 -249
1.04666662216187 -155
1.04999995231628 -91
1.0533332824707 -70
1.05666661262512 -64
1.05999994277954 -60
1.06333339214325 -57
1.06666672229767 -53
1.07000005245209 -38
1.07333338260651 -16
1.07666671276093 9
1.08000004291534 25
1.08333337306976 43
1.08666670322418 59
1.0900000333786 69
1.09333336353302 75
1.09666669368744 81
1.10000002384186 86
1.10333335399628 104
1.1066666841507 135
1.11000001430511 175
1.11333334445953 217
1.11666667461395 262
1.12000000476837 300
1.12333333492279 326
1.12666666507721 339
1.12999999523163 349
1.13333332538605 356
1.13666665554047 355
1.13999998569489 345
1.1433333158493 333
1.14666664600372 321
1.14999997615814 312
1.15333330631256 301
1.15666663646698 287
1.1599999666214 271
1.16333329677582 260
1.16666662693024 244
1.16999995708466 220
1.17333328723907 188
1.17666661739349 155
1.17999994754791 124
1.18333327770233 101
1.18666660785675 76
1.19000005722046 55
1.19333338737488 47
1.1966667175293 41
1.20000004768372 26
1.20333337783813 0
1.20666670799255 -14
1.21000003814697 -28
1.21333336830139 -43
1.21666669845581 -56
1.22000002861023 -72
1.22333335876465 -91
1.22666668891907 -108
1.23000001907349 -124
1.23333334922791 -137
1.23666667938232 -144
1.24000000953674 -149
1.24333333969116 -154
1.24666666984558 -158
1.25 -165
1.25333333015442 -172
1.25666666030884 -167
1.25999999046326 -153
1.26333332061768 -139
1.26666665077209 -134
1.26999998092651 -135
1.27333331108093 -144
1.27666664123535 -151
1.27999997138977 -157
1.28333330154419 -163
1.28666663169861 -167
1.28999996185303 -171
1.29333329200745 -182
1.29666662216187 -183
1.29999995231628 -159
1.3033332824707 -111
1.30666661262512 -49
1.30999994277954 19
1.31333339214325 86
1.31666672229767 150
1.32000005245209 206
1.32333338260651 233
1.32666671276093 226
1.33000004291534 166
1.33333337306976 48
1.33666670322418 -89
1.3400000333786 -226
1.34333336353302 -325
1.34666669368744 -377
1.35000002384186 -405
1.35333335399628 -435
1.3566666841507 -475
1.36000001430511 -519
1.36333334445953 -566
1.36666667461395 -601
1.37000000476837 -605
1.37333333492279 -587
1.37666666507721 -549
1.37999999523163 -508
1.38333332538605 -482
1.38666665554047 -468
1.38999998569489 -447
1.3933333158493 -399
1.39666664600372 -336
1.39999997615814 -265
1.40333330631256 -202
1.40666663646698 -156
1.4099999666214 -122
1.41333329677582 -91
1.41666662693024 -68
1.41999995708466 -52
1.42333328723907 -35
1.42666661739349 -20
1.42999994754791 -5
1.43333327770233 7
1.43666660785675 26
1.44000005722046 49
1.44333338737488 60
1.4466667175293 67
1.45000004768372 74
1.45333337783813 87
1.45666670799255 108
1.46000003814697 125
1.46333336830139 145
1.46666669845581 171
1.47000002861023 182
1.47333335876465 189
1.47666668891907 198
1.48000001907349 216
1.48333334922791 254
1.48666667938232 290
1.49000000953674 308
1.49333333969116 318
1.49666666984558 325
1.5 331
1.50333333015442 337
1.50666666030884 347
1.50999999046326 356
1.51333332061768 365
1.51666665077209 326
1.51999998092651 266
1.52333331108093 217
1.52666664123535 195
1.52999997138977 186
1.53333330154419 176
1.53666663169861 162
1.53999996185303 146
1.54333329200745 121
1.54666662216187 88
1.54999995231628 65
1.5533332824707 55
1.55666661262512 46
1.55999994277954 22
1.56333339214325 -15
1.56666672229767 -48
1.57000005245209 -71
1.57333338260651 -81
1.57666671276093 -85
1.58000004291534 -88
1.58333337306976 -90
1.58666670322418 -91
1.5900000333786 -90
1.59333336353302 -88
1.59666669368744 -85
1.60000002384186 -82
1.60333335399628 -77
1.6066666841507 -78
1.61000001430511 -89
1.61333334445953 -99
1.61666667461395 -110
1.62000000476837 -115
1.62333333492279 -119
1.62666666507721 -124
1.62999999523163 -130
1.63333332538605 -134
1.63666665554047 -137
1.63999998569489 -134
1.6433333158493 -129
1.64666664600372 -125
1.64999997615814 -125
1.65333330631256 -131
1.65666663646698 -140
1.6599999666214 -138
1.66333329677582 -112
1.66666662693024 -52
1.66999995708466 32
1.67333328723907 147
1.67666661739349 284
1.67999994754791 408
1.68333327770233 489
1.68666660785675 519
1.69000005722046 493
1.69333338737488 418
1.6966667175293 315
1.70000004768372 201
1.70333337783813 83
1.70666670799255 -26
1.71000003814697 -111
1.71333336830139 -171
1.71666669845581 -213
1.72000002861023 -227
1.72333335876465 -232
1.72666668891907 -223
1.73000001907349 -213
1.73333334922791 -205
1.73666667938232 -205
1.74000000953674 -210
1.74333333969116 -213
1.74666666984558 -216
1.75 -218
1.75333333015442 -220
1.75666666030884 -221
1.75999999046326 -220
1.76333332061768 -220
1.76666665077209 -218
1.76999998092651 -217
1.77333331108093 -215
1.77666664123535 -212
1.77999997138977 -207
1.78333330154419 -198
1.78666663169861 -188
1.78999996185303 -175
1.79333329200745 -161
1.79666662216187 -149
1.79999995231628 -136
1.8033332824707 -124
1.80666661262512 -116
1.80999994277954 -106
1.81333339214325 -95
1.81666672229767 -81
1.82000005245209 -58
1.82333338260651 -33
1.82666671276093 -5
1.83000004291534 10
1.83333337306976 26
1.83666670322418 41
1.8400000333786 55
1.84333336353302 69
1.84666669368744 75
1.85000002384186 79
1.85333335399628 81
1.8566666841507 83
1.86000001430511 84
1.86333334445953 85
1.86666667461395 85
1.87000000476837 85
1.87333333492279 85
1.87666666507721 83
1.87999999523163 82
1.88333332538605 79
1.88666665554047 75
1.88999998569489 70
1.8933333158493 60
1.89666664600372 43
1.89999997615814 29
1.90333330631256 13
1.90666663646698 -5
1.9099999666214 -20
1.91333329677582 -33
1.91666662693024 -49
1.91999995708466 -70
1.92333328723907 -82
1.92666661739349 -86
1.92999994754791 -88
1.93333327770233 -90
1.93666660785675 -91
1.94000005722046 -94
1.94333338737488 -98
1.9466667175293 -102
1.95000004768372 -106
1.95333337783813 -111
1.95666670799255 -116
1.96000003814697 -120
1.96333336830139 -123
1.96666669845581 -126
1.97000002861023 -130
1.97333335876465 -132
1.97666668891907 -128
1.98000001907349 -120
1.98333334922791 -113
1.98666667938232 -106
1.99000000953674 -100
1.99333333969116 -98
1.99666666984558 -96
2 -94
2.00333333015442 -91
2.00666666030884 -91
2.00999999046326 -97
2.01333332061768 -105
2.01666665077209 -112
2.01999998092651 -118
2.02333331108093 -127
2.02666664123535 -133
2.02999997138977 -124
2.03333330154419 -96
2.03666663169861 -34
2.03999996185303 65
2.04333329200745 194
2.04666662216187 338
2.04999995231628 473
2.0533332824707 545
2.05666661262512 547
2.05999994277954 489
2.06333327293396 354
2.06666660308838 192
2.0699999332428 52
2.07333326339722 -42
2.07666659355164 -98
2.07999992370605 -128
2.08333325386047 -139
2.08666658401489 -145
2.08999991416931 -158
2.09333324432373 -177
2.09666657447815 -183
2.09999990463257 -187
2.10333323478699 -189
2.10666656494141 -190
2.10999989509583 -190
2.11333322525024 -189
2.11666655540466 -187
2.11999988555908 -183
2.1233332157135 -178
2.1266667842865 -169
2.13000011444092 -154
2.13333344459534 -140
2.13666677474976 -126
2.14000010490417 -116
2.14333343505859 -111
2.14666676521301 -105
2.15000009536743 -99
2.15333342552185 -89
2.15666675567627 -75
2.16000008583069 -62
2.16333341598511 -51
2.16666674613953 -36
2.17000007629395 -13
2.17333340644836 13
2.17666673660278 38
2.1800000667572 49
2.18333339691162 55
2.18666672706604 59
2.19000005722046 64
2.19333338737488 70
2.1966667175293 85
2.20000004768372 117
2.20333337783813 145
2.20666670799255 159
2.21000003814697 182
2.21333336830139 203
2.21666669845581 227
2.22000002861023 239
2.22333335876465 246
2.22666668891907 251
2.23000001907349 254
2.23333334922791 251
2.23666667938232 244
2.24000000953674 233
2.24333333969116 215
2.24666666984558 196
2.25 178
2.25333333015442 160
2.25666666030884 145
2.25999999046326 132
2.26333332061768 115
2.26666665077209 94
2.26999998092651 79
2.27333331108093 68
2.27666664123535 55
2.27999997138977 45
2.28333330154419 37
2.28666663169861 27
2.28999996185303 15
2.29333329200745 0
2.29666662216187 -16
2.29999995231628 -31
2.3033332824707 -39
2.30666661262512 -44
2.30999994277954 -47
2.31333327293396 -49
2.31666660308838 -51
2.3199999332428 -50
2.32333326339722 -50
2.32666659355164 -49
2.32999992370605 -47
2.33333325386047 -46
2.33666658401489 -44
2.33999991416931 -43
2.34333324432373 -47
2.34666657447815 -54
2.34999990463257 -58
2.35333323478699 -62
2.35666656494141 -65
2.35999989509583 -68
2.36333322525024 -69
2.36666655540466 -69
2.36999988555908 -68
2.3733332157135 -67
2.3766667842865 -65
2.38000011444092 -62
2.38333344459534 -59
2.38666677474976 -53
2.39000010490417 -50
2.39333343505859 -52
2.39666676521301 -58
2.40000009536743 -67
2.40333342552185 -81
2.40666675567627 -93
2.41000008583069 -101
2.41333341598511 -87
2.41666674613953 -49
2.42000007629395 46
2.42333340644836 185
2.42666673660278 346
2.4300000667572 503
2.43333339691162 608
2.43666672706604 631
2.44000005722046 574
2.44333338737488 452
2.4466667175293 291
2.45000004768372 131
2.45333337783813 3
2.45666670799255 -91
2.46000003814697 -148
2.46333336830139 -165
2.46666669845581 -169
2.47000002861023 -173
2.47333335876465 -176
2.47666668891907 -180
2.48000001907349 -194
2.48333334922791 -215
2.48666667938232 -222
2.49000000953674 -214
2.49333333969116 -187
2.49666666984558 -152
2.5 -129
2.50333333015442 -120
2.50666666030884 -118
2.50999999046326 -121
2.51333332061768 -130
2.51666665077209 -142
2.51999998092651 -161
2.52333331108093 -179
2.52666664123535 -189
2.52999997138977 -182
2.53333330154419 -161
2.53666663169861 -142
2.53999996185303 -135
2.54333329200745 -131
2.54666662216187 -127
2.54999995231628 -115
2.5533332824707 -89
2.55666661262512 -62
2.55999994277954 -40
2.56333327293396 -32
2.56666660308838 -27
2.5699999332428 -24
2.57333326339722 -15
2.57666659355164 18
2.57999992370605 58
2.58333325386047 89
2.58666658401489 112
2.58999991416931 126
2.59333324432373 142
2.59666657447815 158
2.59999990463257 178
2.60333323478699 192
2.60666656494141 201
2.60999989509583 200
2.61333322525024 177
2.61666655540466 149
2.61999988555908 141
2.6233332157135 135
2.6266667842865 131
2.63000011444092 126
2.63333344459534 110
2.63666677474976 82
2.64000010490417 54
2.64333343505859 39
2.64666676521301 30
2.65000009536743 22
2.65333342552185 11
2.65666675567627 -16
2.66000008583069 -56
2.66333341598511 -80
2.66666674613953 -87
2.67000007629395 -93
2.67333340644836 -97
2.67666673660278 -102
2.6800000667572 -119
2.68333339691162 -140
2.68666672706604 -148
2.69000005722046 -149
2.69333338737488 -148
2.6966667175293 -145
2.70000004768372 -142
2.70333337783813 -152
2.70666670799255 -159
2.71000003814697 -164
2.71333336830139 -168
2.71666669845581 -170
2.72000002861023 -171
2.72333335876465 -170
2.72666668891907 -167
2.73000001907349 -164
2.73333334922791 -161
2.73666667938232 -158
2.74000000953674 -155
2.74333333969116 -151
2.74666666984558 -149
2.75 -149
2.75333333015442 -151
2.75666666030884 -154
2.75999999046326 -160
2.76333332061768 -169
2.76666665077209 -181
2.76999998092651 -189
2.77333331108093 -188
2.77666664123535 -165
2.77999997138977 -126
2.78333330154419 -68
2.78666663169861 12
2.78999996185303 113
2.79333329200745 215
2.79666662216187 300
2.79999995231628 332
2.8033332824707 327
2.80666661262512 271
2.80999994277954 185
2.81333327293396 85
2.81666660308838 -13
2.8199999332428 -78
2.82333326339722 -125
2.82666659355164 -199
2.82999992370605 -292
2.83333325386047 -362
2.83666658401489 -386
2.83999991416931 -365
2.84333324432373 -303
2.84666657447815 -228
2.84999990463257 -170
2.85333323478699 -148
2.85666656494141 -148
2.85999989509583 -173
2.86333322525024 -195
2.86666655540466 -203
2.86999988555908 -206
2.8733332157135 -204
2.8766667842865 -196
2.88000011444092 -175
2.88333344459534 -148
2.88666677474976 -121
2.89000010490417 -103
2.89333343505859 -99
2.89666676521301 -96
2.90000009536743 -94
2.90333342552185 -91
2.90666675567627 -88
2.91000008583069 -85
2.91333341598511 -82
2.91666674613953 -66
2.92000007629395 -44
2.92333340644836 -20
2.92666673660278 -6
2.9300000667572 19
2.93333339691162 50
2.93666672706604 84
2.94000005722046 115
2.94333338737488 125
2.9466667175293 131
2.95000004768372 135
2.95333337783813 146
2.95666670799255 169
2.96000003814697 192
2.96333336830139 209
2.96666669845581 229
2.97000002861023 243
2.97333335876465 249
2.97666668891907 253
2.98000001907349 255
2.98333334922791 252
2.98666667938232 245
2.99000000953674 224
2.99333333969116 201
2.99666666984558 188
3 165
3.00333333015442 132
3.00666666030884 100
3.00999999046326 77
3.01333332061768 66
3.01666665077209 52
3.01999998092651 31
3.02333331108093 7
3.02666664123535 -19
3.02999997138977 -30
3.03333330154419 -36
3.03666663169861 -39
3.03999996185303 -34
3.04333329200745 -29
3.04666662216187 -25
3.04999995231628 -24
3.0533332824707 -27
3.05666661262512 -33
3.05999994277954 -41
3.06333327293396 -50
3.06666660308838 -59
3.0699999332428 -68
3.07333326339722 -77
3.07666659355164 -85
3.07999992370605 -93
3.08333325386047 -101
3.08666658401489 -101
3.08999991416931 -94
3.09333324432373 -89
3.09666657447815 -85
3.09999990463257 -82
3.10333323478699 -79
3.10666656494141 -76
3.10999989509583 -73
3.11333322525024 -69
3.11666655540466 -63
3.11999988555908 -54
3.1233332157135 -47
3.1266667842865 -31
3.13000011444092 5
3.13333344459534 70
3.13666677474976 162
3.14000010490417 271
3.14333343505859 369
3.14666676521301 427
3.15000009536743 433
3.15333342552185 409
3.15666675567627 319
3.16000008583069 199
3.16333341598511 98
3.16666674613953 32
3.17000007629395 -7
3.17333340644836 -30
3.17666673660278 -48
3.1800000667572 -86
3.18333339691162 -134
3.18666672706604 -169
3.19000005722046 -182
3.19333338737488 -178
3.1966667175293 -168
3.20000004768372 -151
3.20333337783813 -148
3.20666670799255 -145
3.21000003814697 -144
3.21333336830139 -143
3.21666669845581 -142
3.22000002861023 -140
3.22333335876465 -138
3.22666668891907 -136
3.23000001907349 -134
3.23333334922791 -132
3.23666667938232 -130
3.24000000953674 -127
3.24333333969116 -124
3.24666666984558 -115
3.25 -95
3.25333333015442 -76
3.25666666030884 -60
3.25999999046326 -50
3.26333332061768 -42
3.26666665077209 -34
3.26999998092651 -17
3.27333331108093 6
3.27666664123535 15
3.27999997138977 21
3.28333330154419 29
3.28666663169861 41
3.28999996185303 67
3.29333329200745 96
3.29666662216187 127
3.29999995231628 161
3.3033332824707 173
3.30666661262512 176
3.30999994277954 178
3.31333327293396 179
3.31666660308838 180
3.3199999332428 180
3.32333326339722 180
3.32666659355164 180
3.32999992370605 178
3.33333325386047 176
3.33666658401489 173
3.33999991416931 170
3.34333324432373 162
3.34666657447815 148
3.34999990463257 127
3.35333323478699 99
3.35666656494141 71
3.35999989509583 45
3.36333322525024 20
3.36666655540466 3
3.36999988555908 -10
3.3733332157135 -18
3.3766667842865 -27
3.38000011444092 -35
3.38333344459534 -50
3.38666677474976 -65
3.39000010490417 -75
3.39333343505859 -88
3.39666676521301 -104
3.40000009536743 -114
3.40333342552185 -119
3.40666675567627 -123
3.41000008583069 -115
3.41333341598511 -103
3.41666674613953 -94
3.42000007629395 -88
3.42333340644836 -84
3.42666673660278 -90
3.4300000667572 -95
3.43333339691162 -99
3.43666672706604 -102
3.44000005722046 -107
3.44333338737488 -113
3.4466667175293 -117
3.45000004768372 -121
3.45333337783813 -125
3.45666670799255 -128
3.46000003814697 -129
3.46333336830139 -129
3.46666669845581 -126
3.47000002861023 -121
3.47333335876465 -114
3.47666668891907 -90
3.48000001907349 -40
3.48333334922791 33
3.48666667938232 129
3.49000000953674 241
3.49333333969116 348
3.49666666984558 424
3.5 452
3.50333333015442 428
3.50666666030884 351
3.50999999046326 230
3.51333332061768 99
3.51666665077209 -9
3.51999998092651 -85
3.52333331108093 -134
3.52666664123535 -161
3.52999997138977 -169
3.53333330154419 -173
3.53666663169861 -178
3.53999996185303 -182
3.54333329200745 -185
3.54666662216187 -192
3.54999995231628 -200
3.5533332824707 -201
3.55666661262512 -192
3.55999994277954 -179
3.56333327293396 -164
3.56666660308838 -158
3.5699999332428 -154
3.57333326339722 -151
3.57666659355164 -150
3.57999992370605 -148
3.58333325386047 -147
3.58666658401489 -145
3.58999991416931 -143
3.59333324432373 -140
3.59666657447815 -130
3.59999990463257 -104
3.60333323478699 -78
3.60666656494141 -63
3.60999989509583 -51
3.61333322525024 -42
3.61666655540466 -33
3.61999988555908 -11
3.6233332157135 5
3.6266667842865 12
3.63000011444092 13
3.63333344459534 11
3.63666677474976 6
3.64000010490417 11
3.64333343505859 31
3.64666676521301 49
3.65000009536743 65
3.65333342552185 79
3.65666675567627 90
3.66000008583069 98
3.66333341598511 107
3.66666674613953 129
3.67000007629395 153
3.67333340644836 163
3.67666673660278 169
3.6800000667572 174
3.68333339691162 178
3.68666672706604 177
3.69000005722046 167
3.69333338737488 153
3.6966667175293 137
3.70000004768372 124
3.70333337783813 110
3.70666670799255 84
3.71000003814697 49
3.71333336830139 15
3.71666669845581 -7
3.72000002861023 -14
3.72333335876465 -20
3.72666668891907 -25
3.73000001907349 -30
3.73333334922791 -35
3.73666667938232 -40
3.74000000953674 -45
3.74333333969116 -49
3.74666666984558 -52
3.75 -56
3.75333333015442 -65
3.75666666030884 -76
3.75999999046326 -82
3.76333332061768 -87
3.76666665077209 -92
3.76999998092651 -99
3.77333331108093 -106
3.77666664123535 -113
3.77999997138977 -113
3.78333330154419 -105
3.78666663169861 -100
3.78999996185303 -95
3.79333329200745 -92
3.79666662216187 -90
3.79999995231628 -88
3.8033332824707 -87
3.80666661262512 -85
3.80999994277954 -82
3.81333327293396 -93
3.81666660308838 -107
3.8199999332428 -120
3.82333326339722 -126
3.82666659355164 -130
3.82999992370605 -134
3.83333325386047 -137
3.83666658401489 -137
3.83999991416931 -133
3.84333324432373 -126
3.84666657447815 -105
3.84999990463257 -53
3.85333323478699 31
3.85666656494141 152
3.85999989509583 302
3.86333322525024 440
3.86666655540466 535
3.86999988555908 564
3.8733332157135 512
3.8766667842865 385
3.88000011444092 215
3.88333344459534 47
3.88666677474976 -69
3.89000010490417 -112
3.89333343505859 -128
3.89666676521301 -138
3.90000009536743 -148
3.90333342552185 -164
3.90666675567627 -187
3.91000008583069 -205
3.91333341598511 -212
3.91666674613953 -202
3.92000007629395 -185
3.92333340644836 -169
3.92666673660278 -161
3.9300000667572 -158
3.93333339691162 -155
3.93666672706604 -151
3.94000005722046 -148
3.94333338737488 -145
3.9466667175293 -140
3.95000004768372 -132
3.95333337783813 -119
3.95666670799255 -109
3.96000003814697 -101
3.96333336830139 -95
3.96666669845581 -90
3.97000002861023 -87
3.97333335876465 -83
3.97666668891907 -80
3.98000001907349 -76
3.98333334922791 -73
3.98666667938232 -70
3.99000000953674 -65
3.99333333969116 -47
3.99666666984558 -24
4 1
4.003333568573 14
4.00666666030884 33
4.01000022888184 54
4.01333332061768 80
4.01666688919067 104
4.01999998092651 113
4.02333354949951 120
4.02666664123535 127
4.03000020980835 132
4.03333330154419 136
4.03666687011719 138
4.03999996185303 142
4.04333353042603 146
4.04666662216187 139
4.05000019073486 134
4.0533332824707 130
4.0566668510437 127
4.05999994277954 124
4.06333351135254 121
4.06666660308838 117
4.07000017166138 103
4.07333326339722 76
4.07666683197021 51
4.07999992370605 38
4.08333349227905 28
4.08666658401489 21
4.09000015258789 9
4.09333324432373 0
4.09666681289673 -9
4.09999990463257 -14
4.10333347320557 -12
4.10666656494141 -9
4.1100001335144 -2
4.11333322525024 -6
4.11666679382324 -37
4.11999988555908 -64
4.12333345413208 -77
4.12666654586792 -87
4.13000011444092 -96
4.13333320617676 -103
4.13666677474976 -107
4.1399998664856 -102
4.14333343505859 -89
4.14666652679443 -69
4.15000009536743 -42
4.15333318710327 -35
4.15666675567627 -33
4.15999984741211 -40
4.16333341598511 -65
};
\end{axis}

\end{tikzpicture}

        \caption{ECG signal of the 2017 PhysioNet/CinC Challenge dataset \cite{Clifford2017} labeled as AF.}
        \label{fig:ecg_signal_A}
    \end{figure}
\end{frame}

\subsection{Motivation}
\begin{frame}{Introduction}
\framesubtitle{Motivation}
    \begin{itemize}
        \item Atrial fibrillation (AF) dangerous and often undetected
        \item AF one of the most common heart arrhythmia's
        \item AF can lead to strokes, dementia, and heart failure
        \item Increasing amount of single-lead ECG edge devices available
        \begin{itemize}
            \item No expert knowledge typically available
            \item Need for automated AF
        \end{itemize}
    \end{itemize}
    \begin{textblock*}{3cm}(11.8cm, 3cm)
        \raggedright
        \cite{Becker2006}
        \cite{Herold2019}
    \end{textblock*}
    \pause
    \begin{figure}[!ht] 
        \centering
        \begin{tikzpicture}[line join=round, scale=0.555, >={Stealth[inset=0pt,length=6.0pt,angle'=45]}, every node/.style={font=\fontsize{10}{10}\sffamily}]
	\begin{scope}[x=2pt, y=-2pt]
		\draw[thick] (0,455.0021) -- (11.8345,455.0021);
	        \draw[thick] (11.8345,455.0021) .. controls (14.2834,454.8958) and
	          (14.1385,448.7114) .. (18.8842,448.7114) .. controls (24.2116,448.7114) and
	          (23.9695,454.8958) .. (26.3035,455.0021);
	        \draw[thick] (26.3035,455.0021) -- (40.7723,455.0021);
	        \draw[thick] (40.7723,455.0021) -- (42.7455,465.0645) -- (46.0339,413.3461)
	          -- (48.6647,466.4235) -- (51.2955,455.0021);
	        \draw[thick] (51.2955,455.0021) -- (61.1605,455.0021);
	        \draw[thick] (61.1605,455.0021) .. controls (64.4487,454.3298) and
	          (65.7118,441.6860) .. (70.3679,441.5241) .. controls (75.2428,441.3542) and
	          (76.9447,454.3301) .. (80.2329,455.0021);
	        % \draw[]  (80.2329,455.0021) .. controls (81.4852,455.0021) and
	         %  (82.2677,452.8462) .. (84.1792,452.9863) .. controls (85.8717,453.1101) and
	         % (86.4830,455.0021) .. (87.4678,455.0021);
	        \draw[thick] (80.1678,455.0021) -- (95,455.0021);
	\end{scope}
	\draw[very thick, <->] (0,-28.5) -- (0,-33.52) -- (7,-33.52);
	\node[] at (1.3,-31.2) {P};
	\node[] at (2.9,-33.1) {Q};
	\node[] at (3.2,-28.7) {R};
	\node[] at (3.6,-33.1) {S};
	\node[] at (5,-30.6) {T};
	\node[] at (3.5,-33.91) {time};
	\node[rotate=90] at (-0.4,-30.8) {voltage};
	\begin{scope}[xshift=9.5cm, shift={(-1.54,0)}]
		\begin{scope}[x=2pt, y=-2pt]
			
			\draw[thick] (40.7723,455.0021) -- (42.7455,465.0645) -- (46.0339,413.3461)
			  -- (48.6647,466.4235) -- (51.2955,455.0021);
			% \draw[thick] (51.2955,455.0021) -- (61.1605,455.0021);
			% \draw[thick] (61.1605,455.0021) .. controls (64.4487,454.3298) and
			  (65.7118,441.686) .. (70.3679,441.5241) .. controls (75.2428,441.3542) and
			  (76.9447,454.3301) .. (80.2329,455.0021);
			% \draw[]  (80.2329,455.0021) .. controls (81.4852,455.0021) and
			 %  (82.2677,452.8462) .. (84.1792,452.9863) .. controls (85.8717,453.1101) and
			 % (86.4830,455.0021) .. (87.4678,455.0021);
			% \draw[thick] (80.1678,455.0021) -- (95,455.0021);
		\end{scope}
	    \draw[very thick, <->] (0,-28.5) -- (0,-33.52) -- (7,-33.52);
    	\node[] at (2.9,-33.1) {Q};
    	\node[] at (3.2,-28.7) {R};
    	\node[] at (3.6,-33.1) {S};
	    \node[] at (3.5,-33.91) {time};
		\node[rotate=90] at (-0.4,-30.8) {voltage};
	\end{scope}
	\begin{scope}[xshift=-0.5cm]
	    \draw[thick] (8.46,-32) .. controls (8.8,-32) and (8.8,-31.8) .. (9,-31.8) .. controls (9.2,-31.8) and (9.2,-32) .. (9.4,-31.8) .. controls (9.6,-31.6) and (9.6,-31.8) .. (10,-31.8) .. controls (10.2,-31.8) and (10.4,-32) .. (10.5,-31.7) .. controls (10.6,-31.6) and (10.8,-31.8) .. (11,-32) .. controls (11.1,-32.1) and (11.2,-31.8) .. (11.33,-32);
	    \draw[thick] (12.06,-31.98) .. controls (12.9,-31.7) and (12.7,-31.9) .. (12.9,-32) .. controls (13.4,-31.7) and (13.4,-32) .. (13.7,-31.9) .. controls (13.9,-31.8) and (14,-31.9) .. (14.3,-31.9) .. controls (14.6,-31.9) and (14.6,-32.1) .. (14.7,-32) .. controls (14.8,-31.8) and (15,-31.9) .. (15.1,-32);

	\end{scope}
\end{tikzpicture}
        \vspace{-0.15cm}
        \caption{Regular heart beat left and atrial fibrillation on the right.}
        \label{fig:beat}
    \end{figure}
\end{frame}

\subsection{Related Work}
\begin{frame}{Introduction}
\framesubtitle{Related Work}
    \vspace{-0.5cm}
    \begin{columns}[T]
        \column{.48\textwidth}
            \begin{center}
                \textbf{Transitional ML approaches}
            \end{center}
            \onslide<2->{
            \begin{itemize}
                \item Preprocessing \& Feature extraction
                \begin{itemize}
                    \item Data augmentation
                    \item ECG timing features
                    \item Robust interval features
                    \item Waveform features
                \end{itemize}
                \item Learnable classifier
                \begin{itemize}
                    \item Random forest
                    \item Support vector machines
                    \item XGBoost
                \end{itemize}
                \item \cite{Antink2017, Smisek2017}
            \end{itemize}}
        \column{.01\textwidth}
        \vspace{0.25cm}\rule{.1mm}{.75\textheight}
        \column{.48\textwidth}
            \begin{center}
                \textbf{Deep learning approaches}
            \end{center}
            \onslide<3->{
            \begin{itemize}
                \item Preprocessing
                \begin{itemize}
                    \item Data augmentation
                    \item Data conversion (Spectrogram)
                \end{itemize}
                \item Deep learning classifier
                \item \cite{Zihlmann2017, Mousavi2019, Mashrur2019, Khriji2020, Nonaka2020}
            \end{itemize}}
    \end{columns}
\end{frame}